\begingroup
\begin{table*}[!h]
\centering
\begin{tabular}{|p{3cm}|p{6.9cm}|p{6.4cm}|}
\hline
\textbf{Use Case} & \textbf{In situ reduction via Lagrangian flow map} & \textbf{Post hoc flow analysis} \\
\hline
%\textbf{Task} & Computation of Lagrangian flow map & Computation of particle trajectories for specific analysis \\
%\hline
\textbf{Input / Output} & Simulation vector field / Flow map & Eulerian mesh or flow map / Trajectories \\
\hline
\textbf{Objective} & Store representation of vector field & Analysis and visualization of vector field \\
\hline
\textbf{Vector Field Type} & Unsteady (time-varying) state & Steady and unsteady state \\
\hline
\textbf{Number of Particles} & Depends on sampling strategy and data set size & Depends on analysis task and data set size \\
\hline
\textbf{Number of Steps} & Depends on storage interval/sampling strategy & Depends on analysis task \\
\hline
\textbf{Memory Constraints} & Shared with simulation code and typically limited
%~(trajectory representation and number of time-slices accessible)
%. Consequentially, particle trajectory represention and number of vector field time-slices accessible are restricted 
& Can use all available memory \\
%, but might be shared with visualization software \\
\hline
\textbf{Spatial Decomposition and Data Access} & Simulation-determined with each rank accessing one data block & Can be strategically modified, duplicated, requested on-demand, etc. \\
\hline
\textbf{Communication} & Required every cycle to compute a complete flow map & Can be strategically performed and/or delayed \\
\hline
\textbf{Load Balance} & Depends on sampling strategy & Depends on analysis task \\
\hline
\textbf{Preprocessing} & No prior work & Used in several works \\
\hline
\textbf{Domain Coverage} & Strategy to maintain domain coverage is necessary & Depends on analysis task \\
\hline
\end{tabular}
\caption{Differences in distributed-memory particle advection factors for in situ data reduction and post hoc analysis.}
\label{table:differences}
\end{table*}
\endgroup
