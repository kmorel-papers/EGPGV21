Lagrangian analysis is increasingly being considered as a solution to reduce memory footprint while providing high accuracy for time-varying vector fields under sparse temporal settings.
%
%Theoretically, Lagrangian analysis currently demonstrates state-of-the-art data reduction capabilities for time-varying vector fields under sparse temporal settings.
%
In this paper, we presented a study computing in situ Lagrangian flow maps using GPUs at scale on a modern supercomputer with a focus on scalability.
%
%Computing a Lagrangian flow map involves distributed-memory particle advection, an extensively researched parallel and distributed computing task.
%
We proposed the computation and use of local Lagrangian flow maps and quantitatively evaluated the benefits and limitations of a simple strategy using multiple time-varying vector field data sets.
%
Our study evaluated the impact of the number of particles and test configuration parameters (GPUs, rank, compute nodes) on the total, particle advection, and communication time during in situ analysis.
%
The largest configuration we considered computed 67M particle trajectories over 500 cycles of a $812^{3}$ grid using 2048 GPUs across 512 compute nodes.
%
We empirically showed that in several cases local Lagrangian flow maps can be interpolated with a minimal loss of accuracy for practical configurations while requiring a fraction of the execution time, demonstrating scalability and offering greater predictability for in situ analysis costs.
%
Lastly, we outlined directions for future research for the use of Lagrangian flow maps as an in situ data reduction technique.
