%The major focus of this work is scalability and the use of local Lagrangian flow maps, an intrinsically scalable optimization. 
%
Beyond communication and scalability, several areas for the in situ computation and use of Lagrangian flow maps have not yet been addressed.
%
We list these aspects with an eye toward possible future work.


\textbf{Temporal Sampling.} Prior work has considered only fixed durations, i.e., storage intervals, before particle trajectory locations are stored to disk. 
%
However, the right temporal sampling frequency varies not only from data set to data set but also for each particle trajectory.
%
Future research might look to relax the constraint of storing at fixed intervals, enabling greater temporal resolution of the data for post hoc exploration.
%
%We expect advances in temporal sampling techniques could significantly improve information per byte stored to disk.

\textbf{Spatial Sampling.} Thus far, only uniform sampling has been considered in a distributed memory setting.
%
As techniques explore feature-guided sampling strategies, maintaining load balance across all CNs will increasingly become a challenge.
%
Additionally, sampling can be strategically performed to maintain domain coverage near regions of interest or along boundaries in the case of local flow maps.

\textbf{Trajectory Representation.} In this and prior works, only minimal representations of a trajectory are stored in memory or to disk.
%
Although requiring less memory in situ, the shortcoming of this approach is being limited to linear interpolation between the start and end positions. 
%
Storing multiple locations along a trajectory can improve particle location approximation accuracy using curve-fitting techniques.
%
These approaches, however, could increase the use of runtime memory. 

\textbf{Triggers.} In situ tasks can use triggers to direct analysis and computation efforts. 
%
Triggers can provide the flexibility to react to the underlying vector field by controlling spatial or temporal sampling, communication, and memory usage. 
%The use of triggers to control spatial or temporal sampling, communication, and runtime memory usage can provide techniques with a flexibility to react to the underlying vector field.

%\textbf{Post Hoc Interpolation.} As the extracted Lagrangian flow map increases in complexity, efficiently interpolating it for interactive exploration is difficult.
%%
%For example, interpolating particle trajectories that start and stop at arbitrary times.
%%
%Further, depending on the size of the stored data, distributed Lagrangian-based advection schemes will be needed.
%%

\textbf{Visualization.} 
%As extracted Lagrangian flow maps increase in complexity, efficiently interpolating them for interactive exploration is hard.
%
%For example, interpolating a flow map containing particle trajectories that start and stop at arbitrary times.
%
Depending on the size and format of the stored flow maps, the design of efficient distributed-memory Lagrangian-based advection schemes for interactive exploration is the next step.
%
Further, the current literature lacks qualitative evaluations showing the uncertainty introduced under sparse temporal settings by traditional and Lagrangian-based advection schemes.
%
Another potential direction of research involves direct rendering of the Lagrangian flow maps for time-varying vector field visualization. 
%
