\begingroup
\begin{table}[!h]
\centering
\scalebox{0.9}{
%\resizebox{\columnwidth}{!}{%
\begin{tabular}{|c||c|c|c|c|c|c|}
\hline
\textbf{Test} & \textbf{Interval} & \textbf{Reduction} & \textbf{Particles} & \textbf{Stored} & \textbf{Discarded} \\
\hline
T1 & 5 & \multirow{2}{*}{1:1} & \multirow{2}{*}{4194k} & 99.4\% & 0.6\% \\
T2 & 10 & & & 97.9\% & 2.1\% \\
\hline
T3 & 5 & \multirow{2}{*}{1:8} & \multirow{2}{*}{524k} & 99.6\% & 0.4\% \\
T4 & 10 & & & 98.9\% & 1.1\% \\
\hline
T5 & 5 & \multirow{2}{*}{1:27} & \multirow{2}{*}{155k} & 99.7\% & 0.3\% \\
T6 & 10 & & & 99.1\% & 0.9\% \\
\hline
T7 & 5 & \multirow{2}{*}{1:64} & \multirow{2}{*}{65k} & 99.8\% & 0.2\% \\
T8 & 10 & & & 99.3\% & 0.7\% \\
\hline
\end{tabular}
}
\caption{\fix{Specifications for 8 Jet tests. In most cases, under 2\% of particle trajectories are discarded. Accurate reconstruction of these trajectories~(see Figure~\ref{fig:jet_plot}) was dependent on the absolute number of stored particles.}}
\vspace{-2mm}
\label{jet_tab}
\end{table}
\endgroup
